\documentclass{book}
\usepackage{amsmath, amsthm, thmtools, amssymb}
\usepackage{hyperref, cleveref}
\usepackage[a4paper, margin=2cm]{geometry}

\declaretheorem[
    name=Theorem,
    refname={theorem, theorems},
    Refname={Theorem, Theorems},
    numberwithin=chapter]
    {theorem}
\declaretheorem[
    name=Definition,
    refname={definition, definitions},
    Refname={Definition, Definitions},
    style=definition,
    sibling=theorem]
    {definition}
\declaretheorem[
    style=definition,
    sibling=theorem]
    {examples}
\declaretheorem[
    style=definition,
    numberwithin=examples]
    {example}

\title{Galois Theory}
\author{Ian Stewart, Lukas Innerhofer}
\date{09.11.2020}
\begin{document}
    \maketitle
    \tableofcontents

    \chapter{Classical Algebra}
    \section{Complex Numbers}
    The set of the complex numbers is defined as 
    \(\mathbb{C} = \{a + bi : a,b \in \mathbb{R},i^2 = -1\}\)
    with the following algebraic operations:
    \begin{equation} \label{complex operations}
    \begin{split}
        (a + bi) + (c + di) = (a + c) + (b + d)i \in \mathbb{C}\\
        (a + bi)(c + di) = (ac - bd) + (bc + ad)i \in \mathbb{C}\\
        :a,b,c,d \in \mathbb{R}
    \end{split}
    \end{equation}

    \section{Subfields and Subrings of the Complex Numbers}
    \begin{definition}
        A \textit{subring} of \(\mathbb{C}\) is a subset \(R \subseteq \mathbb{C}\) such that
        \[1 \in R\]
        \[\forall x,y \in R \ x + y,-x,xy \in R\]
        A ring with the condition \(1 \in R\) is often called 'ring-with-1' or 'unital ring'.
        We use ring as an abbreviation.\\
        A \textit{subfield} of \(\mathbb{C}\) is a subring \(K \subseteq \mathbb{C}\)
        with the additional condition
        \[\forall x \in K \ x^{-1} \in K\]
        It follows that \(\forall x \in R \ x - x = 0 \in R\)
        and that R is closed under the algebraic operations of addition, subtraction and multiplication.
        K inherits these attributes and is also closed under division.
    \end{definition}

    \begin{examples}
    \begin{example} \label{ex:subring}
        \(R = \{a + bi : a,b \in \mathbb{Z}\}\) is a subring of \(\mathbb{C}\) but not a subfield.
    \end{example}
    \begin{proof}
        Let \(a,b,c,d \in \mathbb{Z}, x = a + bi, y = c + di \in R\)
        \begin{equation*}
        \begin{split}
            x + y = (a + bi) + (c + di) = (a + c) + (b + d)i \in R\\
            -x = -a - bi \in R\\
            xy = (a + bi)(c + di) = (ac - bd) + (bc + ad)i \in R\\
            2 \in R, 2^{-1} \notin R
        \end{split}
        \end{equation*}
    \end{proof}

    \begin{example}
        \(K = \{a + bi : a,b \in \mathbb{Q}\}\) is a subfield of \(\mathbb{C}\).
    \end{example}
    \begin{proof}
        The same arguments from \cref{ex:subring} hold, except
        \(x^{-1} = \frac{a}{a^2 + b^2} + \frac{b}{a^2 + b^2}i \in K\)
    \end{proof}

    \begin{example}
        \(R = \{\sum_{n = 0}^\infty a_n\pi^n : a_n \in \mathbb{Z}\}\) is a subring of \(\mathbb{C}\) but not a subfield.
    \end{example}
    \begin{proof}
        Let \(x, y \in R\)
        \[ x = \sum_{n = 0}^\infty a_n\pi^n, y = \sum_{n = 0}^\infty b_n\pi^n : a_n,b_n \in \mathbb{Z}\]
        Then
        \begin{equation*}
        \begin{split}
            x + y = \sum_{n = 0}^\infty (a_n + b_n)\pi^n \in R\\
            -x = \sum_{n = 0}^\infty -a_n\pi^n \in R\\
            xy = \sum_{n = 0}^\infty a_n\pi^ny \in R \impliedby
            \forall n \in \mathbb{N} \ a_n\pi^ny = 
            \sum_{m = 0}^\infty b_m\pi^ma_n\pi^n = 
            \sum_{m = 0}^\infty a_nb_m\pi^{m + n} \in R\\
            2 \in R, 2^{-1} \notin R
        \end{split}
        \end{equation*}
    \end{proof}
    \end{examples}
\end{document}