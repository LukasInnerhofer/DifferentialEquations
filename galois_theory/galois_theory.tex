\documentclass{book}
\usepackage{amsmath, amsthm, thmtools, amssymb}
\usepackage{hyperref, cleveref}
\usepackage[a4paper, margin=2cm]{geometry}

\declaretheorem[
    name=Theorem,
    refname={theorem, theorems},
    Refname={Theorem, Theorems},
    numberwithin=chapter
]{theorem}
\declaretheorem[
    name=Proposition,
    refname={proposition, propositions},
    Refname={Proposition, Propositions},
    sibling=theorem
]{proposition}
\declaretheorem[
    name=Definition,
    refname={definition, definitions},
    Refname={Definition, Definitions},
    style=definition,
    sibling=theorem
]{definition}
\declaretheorem[
    style=definition,
    sibling=theorem
]{examples}
\declaretheorem[
    style=definition,
    numberwithin=examples
]{example}

\title{Galois Theory}
\author{Ian Stewart, Lukas Innerhofer}
\date{09.11.2020}
\begin{document}
    \maketitle
    \tableofcontents

    \section{Complex Numbers}
The set of the complex numbers is defined as 
$\mathbb{C} = \{a + bi : a,b \in \mathbb{R},i^2 = -1\}$
with the following algebraic operations:
\begin{equation} \label{equation: complex operations}
\begin{split}
    (a + bi) + (c + di) = (a + c) + (b + d)i \in \mathbb{C}\\
    (a + bi)(c + di) = (ac - bd) + (bc + ad)i \in \mathbb{C}\\
    :a,b,c,d \in \mathbb{R}
\end{split}
\end{equation}

    \chapter{Transcendental Numbers}
    \begin{theorem} \label{theorem: pi is transcendental}
        $\pi$ is transcendental.
    \end{theorem}
\end{document}