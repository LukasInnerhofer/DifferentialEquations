\documentclass{book}
\usepackage{amsmath, amsthm, thmtools, amssymb}
\usepackage{hyperref, cleveref}
\usepackage[a4paper, margin=2cm]{geometry}

\declaretheorem[
    name=Theorem,
    refname={theorem, theorems},
    Refname={Theorem, Theorems},
    numberwithin=chapter
]{theorem}
\declaretheorem[
    name=Proposition,
    refname={proposition, propositions},
    Refname={Proposition, Propositions},
    sibling=theorem
]{proposition}
\declaretheorem[
    name=Definition,
    refname={definition, definitions},
    Refname={Definition, Definitions},
    style=definition,
    sibling=theorem
]{definition}
\declaretheorem[
    style=definition,
    sibling=theorem
]{examples}
\declaretheorem[
    style=definition,
    numberwithin=examples
]{example}

\title{Galois Theory}
\author{Ian Stewart, Lukas Innerhofer}
\date{09.11.2020}
\begin{document}
    \maketitle
    \tableofcontents

    \section{Subfields and Subrings of the Complex Numbers}
\begin{definition}
    A \textit{subring} of $\mathbb{C}$ is a subset $R \subseteq \mathbb{C}$ such that
    \[1 \in R\]
    \[\forall x,y \in R \ x + y,-x,xy \in R\]
    A ring with the condition $1 \in R$ is often called 'ring-with-1' or 'unital ring'.
    We use ring as an abbreviation.\\
    A \textit{subfield} of $\mathbb{C}$ is a subring $K \subseteq \mathbb{C}$
    with the additional condition
    \[\forall x \in K \ x^{-1} \in K\]
    It follows that $\forall x \in R \ x - x = 0 \in R$
    and that $R$ is closed under the algebraic operations of addition, subtraction and multiplication.
    $K$ inherits these attributes and is also closed under division.
\end{definition}

\begin{examples}
\begin{example} \label{example: subring}
    $R = \{a + bi : a,b \in \mathbb{Z}\}$ is a subring of $\mathbb{C}$ but not a subfield.
\end{example}
\begin{proof}
    Let $a,b,c,d \in \mathbb{Z}, x = a + bi, y = c + di \in R$
    \[x + y = (a + bi) + (c + di) = (a + c) + (b + d)i \in R\]
    \[-x = -a - bi \in R\]
    \[xy = (a + bi)(c + di) = (ac - bd) + (bc + ad)i \in R\]
    \[2 \in R, 2^{-1} \notin R\]
\end{proof}

\begin{example}
    $K = \{a + bi : a,b \in \mathbb{Q}\}$ is a subfield of $\mathbb{C}$.
\end{example}
\begin{proof}
    The same arguments from \cref{example: subring} hold, except
    $x^{-1} = \frac{a}{a^2 + b^2} + \frac{b}{a^2 + b^2}i \in K$
\end{proof}

\begin{example} \label{example: polynomials in pi with integer coefficients}
    The set of all polynomials in $\pi$ with integer coefficients
    \[R = \{p(\pi) : p(\pi) = \sum_{n = 0}^N a_n\pi^n,a_n \in \mathbb{Z},n,N \in \mathbb{N},N > 0\}\] 
    is a subring of $\mathbb{C}$ but not a subfield.
\end{example}
\begin{proof}
    Let 
    \begin{equation*}
    \begin{split}
        x, y \in R, 
        x = \sum_{n = 0}^N a_n\pi^n, 
        y = \sum_{m = 0}^M b_m\pi^m
            : a_n,b_m \in \mathbb{Z},\\
        a_k = 0\ \forall k > N,
        b_k = 0\ \forall k > M : k \in \mathbb{N}
    \end{split}
    \end{equation*}
    \[K = max(N, M)\]
    Then
    \[x + y = \sum_{k = 0}^K (a_k + b_k)\pi^k \in R\]
    \[-x = \sum_{n = 0}^N -a_n\pi^n \in R\]
    \[
        xy = \sum_{n = 0}^N a_n\pi^ny \in R \impliedby
        \forall n \in \mathbb{N} \ a_n\pi^ny = 
        \sum_{m = 0}^M b_m\pi^ma_n\pi^n = 
        \sum_{m = 0}^M a_nb_m\pi^{m + n} \in R
    \]
    \[2 \in R, 2^{-1} \notin R\]
\end{proof}

\begin{example} \label{example: polynomials in pi with rational coefficients}
    The set of all polynomials in $\pi$ with rational coefficients
    \[R = \{\sum_{n = 0}^\infty a_n\pi^n : a_n \in \mathbb{Q}\}\]
    is a subring of $\mathbb{C}$ but not a subfield.
\end{example}
\begin{proof}
    For additive, subtractive and multiplicative closure the arguments from \cref{example: polynomials in pi with integer coefficients} hold.
    Suppose $\pi^{-1} = f(\pi)$ where $f$ is a polynomial over $\mathbb{Q}$.
    $\implies \pi f(\pi) - 1 = 0$, i.e. $\pi$ satisfies a nontrivial polynomial equation with rational coefficients,
    contrary to \cref{theorem: pi is transcendental}. 
\end{proof}

\begin{example}
    Let $R$ be the set of all polynomials in $\pi$ with rational coefficients
    as defined in \cref{example: polynomials in pi with rational coefficients}.\\
    The set of all rational expressions of polynomials in $\pi$ with rational coefficients
    \[K = \{\frac{p(\pi)}{q(\pi)} : p,q \in R, q \neq 0\}\]
    is a subfield of $\mathbb{C}$.
\end{example}
\begin{proof}
    Let $x,y \in K,x = \frac{p}{q},y = \frac{r}{s}$.
    \[x + y = \frac{ps + rq}{qs} \in K \impliedby ps + rq,qs \in R,qs \neq 0\]
    \[-x = \frac{-p}{q} \in K\]
    \[xy = \frac{pr}{qs} \in K \impliedby pr,qs \in R,qs \neq 0\]
    \[x^{-1} = \frac{q}{p} \in K : x \neq 0\]
\end{proof}

\begin{example}
    The set of all even integers $2\mathbb{Z}$ is (by our convention) not a ring because $1 \notin 2\mathbb{Z}$.
\end{example}

\begin{example}
    $S = \{a + b\sqrt[3]{2} : a,b \in \mathbb{Q}\}$ is not a ring.
\end{example}
\begin{proof}
    $\sqrt[3]{2}\sqrt[3]{2} \notin S$
\end{proof}
\end{examples}

\begin{definition}
    For the following definitions, let $\phi : K \to L$ be a map or function from set $K$ to set $L$.\\
    An \textit{injective} function pairs each element of the domain with exactly one element of the codomain.
    \[\phi(x) = \phi(y) \implies x = y \quad \forall x,y \in K\]
    A \textit{surjective} function pairs each element of the codomain with at least one element of the domain.
    \[\forall y \in L \ \exists x \in K : \phi(x) = y\]
    A \textit{bijective} function is both injective and surjective. 
    I.e. each element of the domain is paired with exactly one element of the codomain and vice versa.\\
    Suppose $K$ and $L$ are subfields of $\mathbb{C}$.
    A \textit{homomorphism} between $K$ and $L$ is a structure preserving map
    \[\phi(x) : K \to L\]
    \begin{equation*}\label{definition: homomorphism}
        \phi(x + y) = \phi(x) + \phi(y) \quad \phi(xy) = \phi(x)\phi(y) \quad \forall x,y \in K
    \end{equation*}
    An \textit{isomorphism} is a bijective homomorphism.\\
    A \textit{monomorphism} is an injective homomorphism.\\
    An \textit{automorphism} is an isomorphism of $K$ with itself.
\end{definition}

\begin{proposition}
    If $\phi : K \to L$ is an isomorphism, then
    \[\phi(0) = 0\]
    \[\phi(1) = 1\]
    \[\phi(-x) = -\phi(x)\]
    \[\phi(x^{-1}) = \phi(x)^{-1}\] 
\end{proposition}
\begin{proof}
    Let $x \in K$
    \[\phi(x) = \phi(x + 0) = \phi(x) + \phi(0) \implies \phi(0) = 0\]
    \[\phi(x) = \phi(1x) = \phi(1)\phi(x) \implies \phi(1) = 1\]
    \[\phi(x - x) = 0 = \phi(x) + \phi(-x) \implies \phi(-x) = -\phi(x)\]
    \[\phi(xx^{-1}) = 1 = \phi(x)\phi(x^{-1}) \implies \phi(x^{-1}) = \phi(x)^{-1}\]
\end{proof}

\begin{definition}
    $\zeta$ is called a \textit{primitive n}th root of unity iff
    \[\zeta^n = 1 \neq \zeta^m\ \forall m|n\]
    Over $\mathbb{C}$ a standard choice for an $n$th root of unity is $e^{\frac{2\pi i}{n}}$
\end{definition}

\begin{proposition}
    Let $\zeta = e^{\frac{2\pi i}{n}}$. Then $\zeta^k = e^{\frac{2\pi ki}{n}}$ is a primitive $n$th root of unity
    iff $k$ is prime to $n$.
\end{proposition}
\begin{proof}
    We will prove the contrapositive statement: $\zeta^k$ is not a primitive nth root of unity iff $k$ is not prime to $n$.\\
    Let $n = mr, r > 1$. Suppose $\zeta^k$ is not a primitive nth root of unity. Then:
    \[(\zeta^k)^m = \zeta^{mk} = e^{2\pi i\frac{k}{r}} = 1 \iff r|k\]
    So $\exists r > 1$ dividing both $n$ and $k \iff$ $k$ is not prime to $n$.
\end{proof}

\begin{examples}
\begin{example}
    Complex conjugation $a + bi \mapsto a - bi$ is an automorphism of $\mathbb{C}$
\end{example}
\begin{proof}
    Let $x = a + bi,y = c + di,\alpha : \mathbb{C} \to \mathbb{C},a + bi \mapsto a - bi$
    \[\alpha(x + y) = \alpha((a + bi) + (c + di)) = \alpha((a + c) + (b + d)i) = (a + c) - (b + d)i = (a - bi) + (c - di) = \alpha(x) + \alpha(y)\]
    \[\alpha(xy) = \alpha((a + bi)(c + di)) = \alpha((ac - bd) + (bc + ad)i) = (ac - bd) - (bc + ad)i = (a - bi)(c - di) = \alpha(x)\alpha(y)\]
    \[x \neq y \implies \alpha(x) \neq \alpha(y)\]
    \[\forall v \in \mathbb{C},v = a + bi \ \exists u \in \mathbb{C},u = a - bi : \alpha(u) = v\]
\end{proof}
\end{examples}

    \chapter{Transcendental Numbers}
    \begin{theorem} \label{theorem: pi is transcendental}
        $\pi$ is transcendental.
    \end{theorem}
\end{document}