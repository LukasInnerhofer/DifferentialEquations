\documentclass{article}
\usepackage{amsmath, amsthm}
\usepackage{cleveref}
\title{Differential Equations}
\author{Lukas Innerhofer}
\date{09.11.2020}
\begin{document}
    \maketitle

    \section{First Order Inhomogeneous Linear Differential Equations}
    \newtheorem{theorem}{Theorem}[section]
    \begin{theorem}
        Differential equations of the form:
        \begin{equation} \label{differential equation}
            \frac{dy(x)}{dx}+\mu(x)y(x) = \sigma(x)
        \end{equation}
        are called first order inhomogeneous linear differential equations and 
        have the following general solution:
        \begin{equation} \label{general solution}
            y(x) = e^{-\int \mu(x)dx} \int e^{\int \mu(x)dx} \sigma(x) dx
        \end{equation}
    \end{theorem}
    Two proofs of this theorem will be discussed.

    \subsection{Integrating Factor}
    \begin{proof}
        Recall the product rule for derivatives:
        \begin{theorem}[Product Rule]
            \begin{equation} \label{product rule}
                \frac{d}{dx}(uv)=\frac{du}{dx}v+u\frac{dv}{dx}
            \end{equation}
        \end{theorem}
        By multiplying both sides of \cref{differential equation} by a cleverly chosen
        exponential function, the left side can be turned into the derivative of 
        a product.
        \[ e^{\int \mu(x)dx} (\frac{dy(x)}{dx}+\mu(x)y) = e^{\int \mu(x)dx} \sigma(x) \]
        \[ e^{\int \mu(x)dx}\frac{dy}{dx} + \mu(x)e^{\int \mu(x)dx}y = e^{\int \mu(x)dx} \sigma(x) \]
        Thus, by \cref{product rule}:
        \[ \frac{d}{dx}(e^{\int \mu(x)dx}y(x)) = e^{\int \mu(x)dx} \sigma(x) \]
        \[ e^{\int \mu(x)dx}y(x) = \int e^{\int \mu(x)dx} \sigma(x) dx \]
        \[ y(x) = e^{-\int \mu(x)dx} \int e^{\int \mu(x)dx} \sigma(x) dx \]
    \end{proof}

    \subsection{Variation of Constants}
    \begin{proof}
        I can't tell you why we're allowed to do this, but here goes:\\
        We first look at the homogeneous version of \cref{differential equation}.
        \[ \frac{dy(x)}{dx}+\mu(x)y(x) = 0 \]
        \[ \frac{dy(x)}{y(x)} = -\mu(x)dx \]
        \[ \int \frac{dy(x)}{y(x)} = -\int \mu(x)dx \]
        \[ ln(y(x)) + C = -\int \mu(x)dx \]
        \[ y(x) = e^{-\int \mu(x)dx + C} = Ce^{-\int \mu(x)dx} \]
        We now turn the constant of integration into a function of x:
        \begin{equation} \label{y with varying constant}
            y(x) = c(x)e^{-\int \mu(x)dx}
        \end{equation}
        Plugging this into the original \cref{differential equation}, we have:
        \[ \frac{d}{dx}(c(x)e^{-\int \mu(x)dx}) + \mu(x)c(x)e^{-\int \mu(x)dx} = \sigma(x) \]
        \begin{equation*}
        \begin{split}
            \frac{dc(x)}{dx}e^{-\int \mu(x)dx} + c(x)\frac{d}{dx}(e^{-\int \mu(x)dx}) + \mu(x)c(x)e^{-\int \mu(x)dx} =\\
            \frac{dc(x)}{dx}e^{-\int \mu(x)dx} - \mu(x)c(x)e^{-\int \mu(x)dx} + \mu(x)c(x)e^{-\int \mu(x)dx} =\\
            \frac{dc(x)}{dx}e^{-\int \mu(x)dx} = \sigma(x)
        \end{split}
        \end{equation*}
        \[ \frac{dc(x)}{dx} = e^{\int \mu(x)dx}\sigma(x) \]
        \[ c(x) = \int e^{\int \mu(x)dx}\sigma(x) dx \]
        Plugging this into \cref{y with varying constant}, we have:
        \[ y(x) = e^{-\int \mu(x)dx} \int e^{\int \mu(x)dx}\sigma(x)dx \]
    \end{proof}
\end{document}